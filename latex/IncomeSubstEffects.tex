%   ===================================================
%   Decomposing a price change into income effect
%       and substitution effect
%   You must have GNUPLOT installed on your computer
%       and LaTeX must be able to call it (--shell-escape)
%   Author: Patrick Blanchenay
%   ===================================================

\documentclass{standalone}
% =============================================
\usepackage{tikz}
\usepackage{pgfplots}
\pgfplotsset{compat=1.8}
% =============================================
\begin{document}
\begin{tikzpicture}
	% ========= Things to change ======================
	\def\xmax{60} \def\ymax{120} % maximum values for quantity and price on the graph (used to draw the axes) 
	\def\oldprice{4} % old price
	\def\newprice{5} % new price
	\def\utilityfunction{20*x^0.5+y} % utility as a function of x (horizontal good) and y (vertical good), here quasilinear in good 1 (no income effect in good1)
	\def\priceone{4} 
	\def\priceoneprime{5}
	\def\pricetwo{2}
	\def\income{200}
	% ========= Compute those manually ======================
	\def\oldutilitylevel{150} % original level of utility (indirect utility)
	\def\newutilitylevel{140} % new level of utility (indirect utility)
	\def\compensatedincome{220} % compute this manually (income needed under new prices to reach old utility level)
	\def\xoneinitial{25} \def\xtwoinitial{50} % Initial bundle
	\def\xonefinal{16} \def\xtwofinal{60} % Final bundle
	\def\xonecomp{16} \def\xtwocomp{70} % Compensated bundle
	
	% ========= Graph begins here ======================
	\begin{axis}[ scale=1.75, % change the scale of the graph (default = 1)
		axis lines=middle,
		view={0}{90}, % it's a 3D graph that we are viewing from above, to see indiff curves
		xlabel=$x_1$, xlabel style={right },
		 ylabel=$x_2$, ylabel style={above },
		xmin=0,ymin=0,xmax=\xmax,ymax=\ymax,
		enlarge y limits={upper=0.1},
		enlarge x limits={upper=0.1},
		xtick={\xoneinitial,\xonefinal}, % or xtick={0,20,...,\xmax},
		xticklabels={$x_1^{i}$,$x_1^{f}$} , 
		ytick={\xtwoinitial,\xtwofinal,\xtwocomp}, % or xtick={0,20,...,\xmax},
		yticklabels={$x_2^{i}$,$x_2^{f}$,$x_2^{SE}$} , % Comment out to display the numerical values
		clip mode=individual, % clip plots to axes box but not text
		]
		% Axes origin
		\node [below left] (origin) at (axis cs:0,0) {$0$};	
		
		% BUDGET CONSTRAINTS
		\addplot[color=red!40, thick,	domain=0:\income/\priceone] { \income/\pricetwo-\priceone/\pricetwo * x  }
			node[pos=0.95,sloped,above,font=\small] {Old BC}; % Old BC
		\addplot[color=blue!40,	thick, domain=0:\income/\priceoneprime] { \income/\pricetwo-\priceoneprime/\pricetwo * x  }
			node[pos=0.9,sloped,below,font=\small] {New BC}; % New BC
		\addplot[color=purple!40,	dashed,thick, domain=0:\compensatedincome/\priceoneprime] { \compensatedincome/\pricetwo-\priceoneprime/\pricetwo * x  }
			node[pos=0.85,sloped,below,font=\small] {Compensated BC}; % Compensated BC
			
		% INDIFFERENCE CURVES	
		\addplot3 [
			very thick, domain=0:\xmax,y domain=0:\ymax,
			contour/labels=false, % Comment out if you want to display the utility levels
			contour gnuplot={levels={\oldutilitylevel,\newutilitylevel}}, % indifference curves utility levels
			samples=80, % lower means more jagged indifference curves, but faster to display
		] {\utilityfunction };
		
		% Choice bundles
		\draw [ fill] (axis cs:\xoneinitial,\xtwoinitial,0) circle (2pt) node [label=above right:$x^{initial}$]  (i) {}; % initial bundle
		\draw [ fill] (axis cs:\xonefinal,  \xtwofinal,0) circle (2pt) node [label=right:{$x^{final}$} ] (f) {}; % final bundle
		\draw [ fill] (axis cs:\xonecomp,  \xtwocomp,0) circle (2pt) node [label=above right:$x^{SE}$] (SE) {}; % hypothetical compensated bundle
		
		% project bundles on 
		\coordinate (verticalaxis) at (axis cs:0,\ymax); 
		\coordinate (horizontalaxis) at (axis cs:\xmax,0); 
		\draw[thick,gray,dotted] (f -| verticalaxis)  -- (f); %  -- (f |- horizontalaxis) ; omitted because of overlap
		\draw[thick,gray,dotted] (i -| verticalaxis) -- (i) -- (i |- horizontalaxis) ; %
		\draw[thick,gray,dotted] (SE -| verticalaxis) -- (SE) -- (SE |- horizontalaxis); % Project (f)
		
		% EFFECTS for Good 1 (no income effect in this case)
		\draw [>=stealth,->,shorten >= 1pt,shorten <= 1pt,thick] (axis cs: \xoneinitial,20,0) 
			-- (axis cs: \xonefinal,20,0) node [midway,above] {TE1};
		\draw [>=stealth,->,shorten >= 1pt,shorten <= 1pt,thick,green!50!black] (axis cs: \xoneinitial,15,0)
			-- (axis cs: \xonefinal,15,0) node [midway,below] {SE1};
		
		% EFFECTS for Good 2
		\draw [>=stealth,->,shorten >= 1pt,shorten <= 1pt,thick] (axis cs: 2,\xtwoinitial,0) 
			-- (axis cs: 2,\xtwofinal,0) node [midway,right] {TE2};
		\draw [>=stealth,->,shorten >= 1pt,shorten <= 1pt,thick,green!50!black] (axis cs: 10,\xtwoinitial,0)
			-- (axis cs: 10,\xtwocomp,0) node [near start,right] {SE2};
		\draw [>=stealth,->,shorten >= 1pt,shorten <= 1pt,thick,orange!50!black] (axis cs: 6,\xtwocomp,0)
			-- (axis cs: 6,\xtwofinal,0) node [midway,right] {IE2};
	
	\end{axis}
\end{tikzpicture}
% =============================================	
\end{document}