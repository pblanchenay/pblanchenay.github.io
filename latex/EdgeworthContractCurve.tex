%   ===================================================
%	Plot an Edgeworth box, with indifference curves, and the contract curve.
%	-- REQUIRES GNUPLOT INSTALLED.

% 	Utility function for each consumer is arbitrary.
%	The graph also displays three arbitrary intersection points for ICs.
%	The ICs are computed automatically based on the intersection points provided.
% 	
%	The user must compute manually the equation of the contract curve.

%	Author: Patrick Blanchenay
%   ===================================================

\documentclass{standalone}
% =============================================
\usepackage{tikz}               % to draw things
\usepackage{pgfplots}           % to draw plots & curves
\pgfplotsset{compat=1.8}
\usetikzlibrary{calc,arrows}  % to find and define intersections of curves, equilibria...

\tikzset{ % to make dots on the graph
dot/.style = {circle, fill, minimum size=#1,
           inner sep=0pt, outer sep=0pt},
dot/.default = 6pt % size of the circle diameter 
}
% =============================================
\begin{document}

\begin{tikzpicture}[,
%
% 
% =======THINGS TO CHANGE =========== 
declare function={
	UA(\xa,\ya)     =   sqrt(\xa * \ya);      			% Arbitrary A's utility function
	UB(\xb,\yb)     =   (\xb)^(1/3)*(\yb)^(2/3);      % Arbitrary B's utility function
	contractcurve(\xa) =  \Qy*\xa / (2*\Qx - \xa); % Equation of Contract Curve (must be computed manually)
}]

\def\Qx{15} % Total quantity of good x (horizontal good)
\def\Qy{10} % Total quantity of good y (vertical good)

% Points where indifference curves should intersect -- cordinates are in A's frame of reference
\def\xC{1} \def\yC{3}	% Point C (not on the CC)
\def\xD{12} \def\yD{2}	% Point D (not on the CC)
\def\xE{\Qx*0.8} \pgfmathsetmacro\yE{contractcurve(\xE)}	% Point E (on the CC), at 80% of width of box

% ===================================


% =======Graphs happens here =========== 

% Compute utility levels for each consumer, at each intersection point.
\pgfmathsetmacro\UAC{UA(\xC,\yC)} \pgfmathsetmacro\UBC{UB(\Qx-\xC,\Qy-\yC)} 
\pgfmathsetmacro\UAD{UA(\xD,\yD)} \pgfmathsetmacro\UBD{UB(\Qx-\xD,\Qy-\yD)} 
\pgfmathsetmacro\UAE{UA(\xE,\yE)} \pgfmathsetmacro\UBE{UB(\Qx-\xE,\Qy-\yE)} 


% 	================ 
% 	CONSUMER A 
%	================
\begin{axis}[axis equal image=true, % Preserve correct proportion of Edgeworth box
	axis lines=middle,
	view={0}{90}, % to view the 3D graph from above
	xlabel=$x_A$, xlabel style={right },
 	xmin=0,	xmax=\Qx+1, xtick=\empty,  domain=0:\Qx+1,
	 	xticklabel pos=top,
 	ylabel=$y_A$, ylabel style={above },
 	ymin=0,ymax=\Qy+1, ytick=\empty,  y domain=0:\Qy+1,
 	clip mode=individual, % clip plots to axes box but not text
]

	% 	================ POINTS OF INTEREST =============
	% Origins
	\node [label={below left:$O_A$}] (origin) at (axis cs:0,0,0) {};	% origin for A
	\node [label={above right:$O_B$}] (originB) at (axis cs:\Qx,\Qy,0) {};	% origin for B
	
	% max quantities
	\node [label={above right:$\Qx$}] (maxQx) at (axis cs:\Qx,0,0) {};	
	\node [label={above right:$\Qy$}] (maxQy) at (axis cs:0,\Qy,0) {};
	
	% Coordinates at intersection points
	\coordinate (C) at (axis cs:\xC,\yC,0) ;	% point C
	\coordinate (D) at (axis cs:\xD,\yD,0) ;	% point D
	\coordinate (E) at (axis cs:\xE,\yE,0) ;	% point E
	
	
	% 	================ CURVES =============
	% Contract curve, including end points
	\addplot[very thick, green!50!black,domain=0:\Qx,smooth] {contractcurve(x)}; % Contract Curve
	
	% Indifference curves of consumer A, going through points C, D, and E
	\addplot3 [
	  	dashed,
	  	contour gnuplot={levels={\UAC,\UAD,\UAE}},
	  	samples=30, % Lower means faster but more jagged
	  	contour/labels=false, % Comment out if you want to display the utility levels on the ICs
	  	contour/draw color={purple}, % Comment out if you want each IC to be of a different color (from blue to red)
	] {UA(x,y)};

\end{axis}

% 	================ 
% 	CONSUMER B
%	================
\begin{axis}[axis equal image=true, % Preserve correct proportion of Edgeworth box
	axis lines=middle,
	view={0}{90}, % to view the 3D graph from above
    anchor=north east, % Shift the axis so its origin is at (originB)
	at={(originB)}, 
    rotate around={180:(current axis.origin)}, % Rotate around its origin
	xlabel=$x_B$, xlabel style={left }, xtick=\empty,
   	xmin=0,	xmax=\Qx+1,  domain=0:\Qx+1,
   	ylabel=$y_B$, ylabel style={below }, ytick=\empty,
   	ymin=0,ymax=\Qy+1,  y domain=0:\Qy+1,
 ]
 
	% Indifference curves of consumer B, going through points C, D, and E
	\addplot3 [
	  	dashed,
	  	contour gnuplot={levels={\UBC,\UBD,\UBE}},
	  	samples=30, % Lower means faster but more jagged
	  	contour/labels=false, % Comment out if you want to display the utility levels on the ICs
	  	contour/draw color={blue}, % Comment out if you want each IC to be of a different color (from blue to red)
	] {UB(x,y)};
\end{axis}

% Draw intersection points, and contract curve end points at the end, so they appear on top of curves
\node [label={above right:$C$},dot]  at (C) {};	% point C
\node [label={above right:$D$},dot]  at (D) {};	% point D
\node [label={right:$E$},dot]  at (E) {};	% point E

% Contract curve end points
\node[dot,green!50!black] at (origin) {};
\node[dot,green!50!black] at (originB) {};

\end{tikzpicture} 


% =============================================	
\end{document}